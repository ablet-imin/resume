\section{Professional Research Experience}

\begin{cventries}
    \multiline{2021 - Present}{Postdoctoral  Associate, New York University}{ }
\end{cventries}
      \vspace{-1cm}
      \parbox{0.9\linewidth}{
      \leftskip=0.5in
       I am actively engaged in cutting-edge research spanning various aspects of particle physics. My recent work has focused on the investigation of Lorentz and CPT Symmetry Violation using the Drell-Yan process, employing a novel approach that incorporates sidereal time-dependent ATLAS luminosity data.

Furthermore, my research portfolio includes pioneering efforts in the application of Machine Learning techniques to enhance the performance of missing ET trigger algorithms for the upcoming High Luminosity LHC (Phase II upgrade). Additionally, I have carried out the study of Event-Filter (EF) level tracking and its impact on Phase II Missing ET triggers, providing valuable insights into improving their overall efficiency and precision.

As a member of the editorial board for dijet analysis with quark-gluon tagging, I am contributing to shaping the direction of this critical research area. I have also served in two expert review roles, offering invaluable guidance and expertise to further enhance resonance search in 3/4-body invariant masses and Mono-S(bb) analyses.\newline
Within the Exotics group, I have taken on the role of Monte-Carlo manager.

        }
 
 \begin{cventries}   
    \multiline{2017 - 2021}{Postdoctoral Research Associate, Argonne National Lab, USA}{ }
  \end{cventries}
  \vspace{-1cm}
   \parbox{0.9\linewidth}{
      \leftskip=0.5in
       I led the development of the multi-b-jet analysis and introduced an innovative data-driven 
       background estimation method. Within the Standard Model group, I proposed a novel time-odd
       asymmetry measurement. My contributions extended to the FELIX phase-1 upgrade project, where
       I played a pivotal role in development and commissioning. Additionally, I ensured the smooth
       operation of the ATLAS PC-based Region of Interest Builder (PC-RoIB) as part of the DAQ group. \newline
        \textbf{Software development}: Developed and utilized common-smoothing tools in the Exotics group.
        I have designed and implemented control software tailored for the FELIG custom board and firmware,
        serving as a crucial resource utilized by numerous sub-detector groups. \newline
        \textbf{R$\&$D projects}: Engaged in R$\&$D efforts focused on Quantum algorithms applicable to high-energy physics.
        Additionally, I have contributed to uncertainty estimations for the neural network-based (DL1) b-tagger. \newline
        \textbf{Service work}: I provided statistical expertise through the HistFitter shifter. 
        Furthermore, I have served as  ATLAS Run-Control and ROS on-call shifter.
    }
    

\begin{cventries}
    \multiline{2017 - 2017}{Research Engineer, Stockholm University}{ }
\end{cventries}
    \vspace{-1cm}
    \parbox{0.9\linewidth}{
      \leftskip=0.5in
        Measure and monitor electronic noises of the tile calorimeter, and update noise parameters in the  ATLAS database.
    }
\begin{cventries}
     \multiline{2011 - 2016}{PhD Researches, Stockholm University}{ }
\end{cventries}
    \vspace{-1cm}
     \parbox{0.9\linewidth}{
      \leftskip=0.5in
        I conducted extensive searches for supersymmetric top quarks using ATLAS data 
        from both Run 1 and Run 2, pioneering the development and optimization of innovative
        analysis strategies. Additionally, I played a pivotal role in constructing 
        3rd generation squarks enriched pMSSM models. My expertise extended to interpreting
        the Run 1 analysis results within the context of the pMSSM models. I maintained an active
        role in contributing to ATLAS data taking.

      }
