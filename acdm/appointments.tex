\section{Professional Appointments and Researches}

\begin{cventries}
    \multiline{2021/09 -- Present}{Postdoctoral  Associate, New York University}{ }
\end{cventries}

       Research and resposibilities in the ATLAS experiment: \newline
       Search for Lorentz and  CPT Symmetry Violation in the Drell-Yan process using 
       sidereal time-dependent ATLAS luminosity. Machin learning approach for missing 
       transverse momentum trigger algorithms for the High Luminosity LHC. 
        Uncertainty estimations of a neural network-based (DL1) b-tagging algorithm.\newline
        Member of the editorial Board of dijet analysis with quark-gluon tagging. 
        An expert review of the dijet analysis. \newline
        Monte-Carlo manager in Exotics group.  \newline
 
 \begin{cventries}   
    \multiline{2017 - 2021}{Postdoctoral Research Associate, Argonne National Lab, USA}{ }
  \end{cventries}
       Developed the multi-b-jet analysis; implemented an innovative data-driven background
       estimation method. Proposed a time-odd asymmetry measurement in the Standard Model group. \newline
        Contributed to the FELIX phase-1 upgrade project; maintained the ATLAS PC-based Region of 
        Interest Builder (PC-RoIB) in the DAQ group. \newline
        Software development: common-smoothing tools and FELIG software packages. \newline
        R$\&$D projects: Quantum algorithms for high energy physics; Uncertainty estimations 
        of the neural network-based (DL1) b-tagger. \newline
        Service work: Provide statistical expertise in the HistFitter. ATLAS Runcontrol shifts,
        ROS on-call shifts. \newline

    

\begin{cventries}
    \multiline{2017 -- 2017}{Research Engineer, Stockholm University}{ }
\end{cventries}
        Measure and monitor electronic noises of the tile calorimeter, and update noise parameters in the  ATLAS database.
 
\begin{cventries}
     \multiline{2011 -- 2016}{PhD Researches, Stockholm University}{ }
\end{cventries}
        Searched supersymmetric top quarks with Run 1 and Run 2 ATLAS data; Develop and optimized analysis strategies, model building, simulation, statistical inference. \newline 
        Interpreted Run 1 analysis results on the pMSSM models. \newline
        Actively contribute to ATLAS detector operations and data taking. \newline
        Teach physics lab courses. \newline
