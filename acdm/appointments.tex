\section{Professional Research Experience}

\begin{cventries}
    \multiline{2021 - Present}{Postdoctoral  Associate, New York University}{ }
\end{cventries}
      \vspace{-1cm}
      \parbox{0.9\linewidth}{
      \leftskip=0.5in
       Search for Lorentz and  CPT Symmetry Violation in the Drell-Yan process using 
       sidereal time-dependent ATLAS luminosity. Machin learning approach for missing transverse momentum (MET) trigger algorithms for the High Luminosity LHC (Phase II upgrade). Studies of the impact of Event-Filter (EF) level tracking on Phase II MET triggers.\newline
        %Uncertainty estimations of a neural network-based (DL1) b-tagging algorithm.\newline
        Member of the editorial Board of dijet analysis with quark-gluon tagging. 
        Played an expert review  role in the dijet analysis. \newline
        Monte-Carlo manager in the Exotics group.
        }
 
 \begin{cventries}   
    \multiline{2017 - 2021}{Postdoctoral Research Associate, Argonne National Lab, USA}{ }
  \end{cventries}
  \vspace{-1cm}
   \parbox{0.9\linewidth}{
      \leftskip=0.5in
       Developed the multi-b-jet analysis; implemented an innovative data-driven background
       estimation method. Proposed a time-odd asymmetry measurement in the Standard Model group. \newline
        Contributed to the FELIX phase-1 upgrade project; maintained the ATLAS PC-based Region of 
        Interest Builder (PC-RoIB) in the DAQ group. \newline
        \textbf{Software development}: common-smoothing tools and FELIG software packages. \newline
        \textbf{R$\&$D projects}: Quantum algorithms for high energy physics; Uncertainty estimations 
        of the neural network-based (DL1) b-tagger. \newline
        \textbf{Service work}: Provide statistical expertise in the HistFitter. ATLAS Run-Control shifts,
        ROS on-call shifts.
    }
    

\begin{cventries}
    \multiline{2017 - 2017}{Research Engineer, Stockholm University}{ }
\end{cventries}
    \vspace{-1cm}
    \parbox{0.9\linewidth}{
      \leftskip=0.5in
        Measure and monitor electronic noises of the tile calorimeter, and update noise parameters in the  ATLAS database.
    }
\begin{cventries}
     \multiline{2011 - 2016}{PhD Researches, Stockholm University}{ }
\end{cventries}
    \vspace{-1cm}
     \parbox{0.9\linewidth}{
      \leftskip=0.5in
        Searched supersymmetric top quarks with Run 1 and Run 2 ATLAS data; Develop and optimized analysis strategies, model building and simulation, statistical inference. \newline 
        Interpreted Run 1 analysis results on the pMSSM models. \newline
        Actively contribute to ATLAS detector operations and data taking. \newline
        Teach physics lab courses.
      }
