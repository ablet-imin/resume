%%%%%%%%%%%%%%%%%%%%%%%%%%%%%%%%%%%%%%%%%%%%%%%%%%%%%%%%%%%%%%%%%%%%%%%%%%%%%%%
% A clean template for an academic CV
%
% Uses tabularx to create two column entries (date and job/edu/citation).
% Defines commands to make adding entries simpler.
%
%%%%%%%%%%%%%%%%%%%%%%%%%%%%%%%%%%%%%%%%%%%%%%%%%%%%%%%%%%%%%%%%%%%%%%%%%%%%%%%

\documentclass[10pt, a4paper]{article}

% Full Unicode support for non-ASCII characters
\usepackage[utf8]{inputenc}

% Identifying information
\newcommand{\Title}{Curriculum Vit\ae}
\newcommand{\FirstName}{Yiming}
\newcommand{\LastName}{Abulaiti}
\newcommand{\Initials}{Y}
\newcommand{\MyName}{Dr. \FirstName\ \LastName}
\newcommand{\Me}{\textbf{\LastName, \Initials}}  % For citations
\newcommand{\Email}{yiming.abulaiti@cern.ch}
\newcommand{\PersonalWebsite}{www.leouieda.com}
\newcommand{\LabWebsite}{https://atlas.cern.ch}
\newcommand{\ORCID}{0000-0003-0403-3697}
\newcommand{\Address}{
 Department of Physics\\
New York University\\
New York, NY 10003
}

% Template configuration
%%%%%%%%%%%%%%%%%%%%%%%%%%%%%%%%%%%%%%%%%%%%%%%%%%%%%%%%%%%%%%%%%%%%%%%%%%%%%%%

% Disable hyphenation
\usepackage[none]{hyphenat}

% Control the font size
\usepackage{anyfontsize}

% Icon fonts (requires using xelatex or luatex)
\usepackage[fixed]{fontawesome5}
\usepackage{academicons}

% Template variables for styling
\newcommand{\TablePad}{\vspace{-0.4cm}}
\newcommand{\SoftwareTitle}[1]{{\bfseries #1}}
\newcommand{\TableTitle}[1]{{\fontsize{12pt}{0}\selectfont \itshape #1}}

% For fancy and multipage tables
\usepackage{tabularx}
\usepackage{ltablex}

% Define a new environment to place all CV entries in a 2-column table.
% Left column are the dates, right column the entries.
\usepackage{environ}
\NewEnviron{EntriesTable}{
\TablePad
\begin{tabularx}{\textwidth}{@{}p{0.10\textwidth}@{\hspace{0.02\textwidth}}p{0.88\textwidth}@{}}
  \BODY
\end{tabularx}
}

% Entries
\newcommand{\entriespad}{0.75em}
\NewEnviron{cventries}{
    \vspace{-1em}
    \begin{tabularx}{\textwidth}{p{0.12\textwidth} p{0.82\textwidth}}
    \BODY
    \end{tabularx}
}

% Entries extended
\NewEnviron{cventriesExtended}{
    \vspace{-1em}
    \begin{tabularx}{\textwidth}{p{0.24\textwidth} p{0.76\textwidth}}
    \BODY
    \end{tabularx}
}

\newcommand{\paper}[2]{%
    {#1}
    &
    {#2}
    \vspace{\entriespad} \\
}

%\newcommand{\singleline}[2]{{#1} & {#2} \vspace{\entriespad} \\}
\newcommand{\singleline}[2]{{#1} & {#2} \\}
\newcommand{\multiline}[3]{{#1} & {\textbf{#2} \newline {#3}} \vspace{0.3em} \\}

% Macros to add links and mark publications
\newcommand{\DOI}[1]{doi:\href{https://doi.org/#1}{#1}}
\newcommand{\DOILink}[1]{\href{https://doi.org/#1}{doi.org/#1}}
\newcommand{\Website}[1]{\href{https://#1}{#1}}
\newcommand{\Preprint}[1]{\newline Preprint: \faFilePdf\ \DOILink{#1}}
\newcommand{\Youtube}[1]{\newline Recording: \faYoutube\, \href{https://www.youtube.com/watch?v=#1}{youtube.com/watch?v=#1}}
\newcommand{\GitHub}[1]{\newline Code: \faGithub\ \href{https://github.com/#1}{github.com/#1}}
\newcommand{\Role}[1]{\newline Role: \faUsers\ #1}
\newcommand{\Slides}[1]{\newline Slides: \faTv\ \href{https://#1}{#1}}
\newcommand{\SlidesDOI}[1]{\newline Slides: \faTv\ \DOILink{#1}}
\newcommand{\PosterDOI}[1]{\newline Poster: \faImage\ \DOILink{#1}}
\newcommand{\OA}{\aiOpenAccess\enspace}

% Macros to set the year and duration on the left column
\newcommand{\Duration}[2]{\fontsize{9pt}{0}\selectfont #1 -- #2}
\newcommand{\Year}[1]{\fontsize{9pt}{0}\selectfont #1}
\newcommand{\Ongoing}{on}
\newcommand{\Future}{future}
\newcommand{\Appointment}[4]{\textbf{#1} \newline #2 \newline #3 \newline #4}

% Define command to insert month name and year as date
\usepackage{datetime}
\newdateformat{monthyear}{\monthname[\THEMONTH], \THEYEAR}

% Set the page margins
\usepackage[a4paper,margin=1.5cm,includehead,headsep=5mm]{geometry}

% To get the total page numbers (\pageref{LastPage})
\usepackage{lastpage}

% No indentation
\setlength\parindent{0cm}

% Increase the line spacing
\renewcommand{\baselinestretch}{1.2}
% and the spacing between rows in tables
\renewcommand{\arraystretch}{1.5}

% Remove space between items in itemize and enumerate
\usepackage{enumitem}
\setlist{nosep}

% Use custom colors
\usepackage[usenames,dvipsnames]{xcolor}

% Set the spacing for sections
\usepackage{titlesec}
\titleformat{\section}
  {\normalfont\Large\mdseries} % format
  {} % label
  {0pt} % separation (left separation for hang)
  {} % text before title
  [\titlerule] % text after title
\titleformat{\subsection}
  {\normalfont\large\mdseries} % format
  {} % label
  {0pt} % separation (left separation for hang)
  {} % text before title

% Disable number of sections. Use this instead of "section*" so that the sections still
% appear as PDF bookmarks. Otherwise, would have to add the table of contents entries
% manually.
\makeatletter
\renewcommand{\@seccntformat}[1]{}
\makeatother

% Set fancy headers
\usepackage{fancyhdr}
\pagestyle{fancy}
\fancyhf{}
\lhead{\fontsize{9pt}{10pt}\selectfont
  \monthyear\today
}
\chead{
  \fontsize{9pt}{10pt}\selectfont
  \MyName
  \hspace{0.2cm} -- \hspace{0.2cm}
  \Title
}
\rhead{\fontsize{9pt}{10pt}\selectfont \thepage{} of \pageref*{LastPage}}
\renewcommand{\headrulewidth}{0pt}

% Metadata for the PDF output and control of hyperlinks
\usepackage[colorlinks=true]{hyperref}
\hypersetup{
  pdftitle={\MyName\ - \Title},
  pdfauthor={\MyName},
  linkcolor=blue,
  citecolor=blue,
  filecolor=black,
  urlcolor=MidnightBlue
}


%%%%%%%%%%%%%%%%%%%%%%%%%%%%%%%%%%%%%%%%%%%%%%%%%%%%%%%%%%%%%%%%%%%%%%%%%%%%%%%

\begin{document}

% No header for the first page
\thispagestyle{empty}

%%%%%%%%%%%%%%%%%%%%%%%%%%%%%%%%%%%%%%%%%%%%%%%%%%%%%%%%%%%%%%%%%%%%%%%%%%%%%%%
\begin{minipage}[t]{0.7\textwidth}
{\fontsize{22pt}{0}\selectfont\MyName}
\end{minipage}
\begin{minipage}[t]{0.3\textwidth}
  \begin{flushright}
    Last updated: \monthyear\today
  \end{flushright}
\end{minipage}
\\[-0.1cm]
\rule{\textwidth}{2pt}
\\[0.1cm]
\begin{minipage}[t]{0.7\textwidth}
    ORCID: \href{https://orcid.org/\ORCID}{\ORCID}
    \\
    Email: \href{mailto:\Email}{\Email}
    \\
    LHC experiment: \Website{\LabWebsite}
    \\
    %Website: \Website{\PersonalWebsite}
\end{minipage}
\begin{minipage}[t]{0.3\textwidth}
  \begin{flushright}
    \Address
  \end{flushright}
\end{minipage}

%%%%%%%%%%%%%%%%%%%%%%%%%%%%%%%%%%%%%%%%%%%%%%%%%%%%%%%%%%%%%%%%%%%%%%%%%%%%%%
%%Summary
\section{Research Interests}
Innovative and collaborative experimental physicist with 10+ years of experience 
searching beyond Standard Model physics at the Large Hadron Collider (LHC). 
Searching for new particles at the LHC is a collaborative work that 
provides an opportunity to lead and work with diverse teams, resulting in the discovery
of the Higgs Boson in July 2012. In the collaboration, I have searched Supersymmetry 
and Exotic particles with a variety of techniques, as well as contributed to detector 
upgrades, published scientific papers, and presented results at major physics conferences.
I am also interested in Dark Matter searches and non-collider experiments.
I conduct research in collaborative and independent environments,  apply Machine Learning
methods to physics research, and have excellent personal and social skills. 


%%%%%%%%%%%%%%%%%%%%%%%%%%%%%%%%%%%%%%%%%%%%%%%%%%%%%%%%%%%%%%%%%%%%%%%%%%%%%%
%%Education
\section{Education}

\begin{cventries}
  \multiline{2011--2016}{PhD in Physics}{
        Department of Physics, Stockholm University, Sweden \\
        %Thesis: Search for Pair-Produced Supersymmetric Top Quark Partners \\ with the ATLAS Experiment \\
		%Supervisors: Dr. Sara Strandberg, Prof. Kerstin Jon-And
    }

  \multiline{2008--2010}{MSc in Physics}{
        Department of Physics, Università di Camerino, Italy \\
        %Thesis: Phenomenology of new particles, their decays \\ and methods of discovery at the LHC \\
	    %Supervisors: Prof. Y.Srivastava, Prof. M.Biasini, University

    }
  
  \multiline{2002--2007}{BSc in Physics}{
        Xinjiang University, Urumqi, China
    }
\end{cventries}


%%%%%%%%%%%%%%%%%%%%%%%%%%%%%%%%%%%%%%%%%%%%%%%%%%%%%%%%%%%%%%%%%%%%%%%%%%%%%%
%%Professonal Appointments
\section{Professional Appointments and Researches}

\begin{cventries}
    \multiline{2021/09 -- Present}{Postdoctoral  Associate, New York University}{ }
\end{cventries}

       Research and resposibilities in the ATLAS experiment: \newline
       Search for Lorentz and  CPT Symmetry Violation in the Drell-Yan process using 
       sidereal time-dependent ATLAS luminosity. Machin learning approach for missing 
       transverse momentum trigger algorithms for the High Luminosity LHC. 
        Uncertainty estimations of a neural network-based (DL1) b-tagging algorithm.\newline
        Member of the editorial Board of dijet analysis with quark-gluon tagging. 
        An expert review of the dijet analysis. \newline
        Monte-Carlo manager in Exotics group.  \newline
 
 \begin{cventries}   
    \multiline{2017 - 2021}{Postdoctoral Research Associate, Argonne National Lab, USA}{ }
  \end{cventries}
       Developed the multi-b-jet analysis; implemented an innovative data-driven background
       estimation method. Proposed a time-odd asymmetry measurement in the Standard Model group. \newline
        Contributed to the FELIX phase-1 upgrade project; maintained the ATLAS PC-based Region of 
        Interest Builder (PC-RoIB) in the DAQ group. \newline
        Software development: common-smoothing tools and FELIG software packages. \newline
        R&D projects: Quantum algorithms for high energy physics; Uncertainty estimations 
        of the neural network-based (DL1) b-tagger. \newline
        Service work: Provide statistical expertise in the HistFitter. ATLAS Runcontrol shifts,
        ROS on-call shifts. \newline

    

\begin{cventries}
    \multiline{2017 -- 2017}{Research Engineer, Stockholm University}{ }
\end{cventries}
        Measure and monitor electronic noises of the tile calorimeter, and update noise parameters in the  ATLAS database.
 
\begin{cventries}
     \multiline{2011 -- 2016}{PhD Researches, Stockholm University}{ }
\end{cventries}
        Searched supersymmetric top quarks with Run 1 and Run 2 ATLAS data; Develop and optimized analysis strategies, model building, simulation, statistical inference. \newline 
        Interpreted Run 1 analysis results on the pMSSM models. \newline
        Actively contribute to ATLAS detector operations and data taking. \newline
        Teach physics lab courses. \newline


%%%%%%%%%%%%%%%%%%%%%%%%%%%%%%%%%%%%%%%%%%%%%%%%%%%%%%%%%%%%%%%%%%%%%%%%%%%%%%
%%Skills
\section{Technical and Personal skills}

\begin{cventriesExtended}
    \singleline{Languages: }{Uyghur (Mothertongue), English (professional), Chinese (professional)}
    \singleline{Computer system:}{Linux, Mac OS, Windows.}
    \singleline{Programming Languages:}{C++, python, ROOT, C, shell scripts.}
    \singleline{Machine leaning tools:}{sklearn, pytorch, TensorFlow-keras,}
    \singleline{Python tools: }{python-notebook, Pandas, numpy, pystats}
    \singleline{Communication skills:}{Over 400 oral presentations at various ATLAS meetings, and conference talks.}

\end{cventriesExtended}


%%%%%%%%%%%%%%%%%%%%%%%%%%%%%%%%%%%%%%%%%%%%%%%%%%%%%%%%%%%%%%%%%%%%%%%%%%%%%%
%%Teaching
\section{Teaching}

\begin{cventries}
    
    \multiline{2016}{Electromagnetism and Waves - Demonstrations}{
        Stockholm University, Sweden.
    }
    \multiline{2013-2015}{Quantum experiments}{
        Stockholm University, Sweden.
    }
    \multiline{2013 -- 2014}{Digital system construction I: FPGA design and programming}{
        Lab assistant \newline
        Stockholm University, Sweden.
    }

\end{cventries}


%%%%%%%%%%%%%%%%%%%%%%%%%%%%%%%%%%%%%%%%%%%%%%%%%%%%%%%%%%%%%%%%%%%%%%%%%%%%%%
%%Outreach
\section{Outreach}

\begin{cventries}
    \singleline{2019}{CERN open day, Geneva, Switzerland}
    \singleline{2013,2015}{Physics event in Kungsträdgården, Stockholm, Sweden.}
\end{cventries}


%%%%%%%%%%%%%%%%%%%%%%%%%%%%%%%%%%%%%%%%%%%%%%%%%%%%%%%%%%%%%%%%%%%%%%%%%%%%%%
%%Grants and Awards
\section{Grants and Awards}

\begin{cventries}
    \singleline{2017 -- 2022}{PhD Scholarship from CONICET}
    \singleline{2019}{Argonne National Lab funding, Develop Quantum Algorithms to simulate higher order LHC Interactions.}
    \singleline{2016}{Liljevalchs travel grant, Stockholm University, Sweden.}
    \singleline{2009-2010}{Scholarship for excellent student, Camerino University, Italy.}
    \singleline{2005-2006}{Jimin Cha Scholarship, Xinjiang University, China.}
    \singleline{2003-2004}{Best Undergraduate Student Award, Xinjiang University, China.}


\end{cventries}


%%%%%%%%%%%%%%%%%%%%%%%%%%%%%%%%%%%%%%%%%%%%%%%%%%%%%%%%%%%%%%%%%%%%%%%%%%%%%%
%%Conferences and Workshops
\section{Conferences, Workshops, Schools and Talks}

\begin{cventries}
    \multiline{2021}{$32^{nd}$ Rencontres de Blois, France}{
        Invited presentation: Status of searches for dark matter at the LHC.
    }
    
    \multiline{2019}{ATLAS Exotic+HDBS workshop, Napoli, Italy}{
         Invited presentation: High mass resonances in hadronic final states: status and plan.
    }
    
    \multiline{2019}{Moriond QCD 2019,  La Thuile, Aosta Valley, Italy}{
        Invited presentation: Searches for New Resonances (lepton, jets) at ATLAS and CMS.
    }
    \singleline{2019}{School of Trigger and Data AcQuisition (ISOTDAQ2019) at Royal Holloway, University of London}
    \singleline{2019}{ATLAS Standard Model workshop, LAL National Lab, France}
     \singleline{2016}{Summer School and Workshop on the Standard Model and Beyond, Corfu, Greece}
     \singleline{2016}{ATLAS SUSY workshop, Sussex, UK.}
     \singleline{2015}{Partikeldagarna national conference, Uppsala, Sweden.}
     \singleline{2015}{CERN HEP summer school, CERN, Geneva, Switzerland.}
     \singleline{2015}{Excellence in Detectors and Instrumentation Technologies (EDIT) school, INFN-LNF, Italy.}

     \multiline{2014}{ICHEP, Valencia, Spain}{
         Poster presentation: Search for direct top squark pair production in final states with one isolated lepton with the ATLAS detector.
     }

     \multiline{2014}{Partikeldagarna national conference, Växjö, Sweden.}{
         Presentation: Search for direct top squark pair production in final states with one isolated lepton with the ATLAS detector.
     }

     \singleline{2014}{ATLAS SUSY $^{3rd}$ generation squarks workshop, Milano, Italy.}

     \singleline{2012}{CHIPP PhD Winter School, Engelberg, Switzerland.}
     \singleline{2012}{Partikeldagarna national conference, Stockholm, Sweden.}



\end{cventries}


%%%%%%%%%%%%%%%%%%%%%%%%%%%%%%%%%%%%%%%%%%%%%%%%%%%%%%%%%%%%%%%%%%%%%%%%%%%%%%
%%References
\newpage
\section{References}

\begin{cventriesExtended}
    \multiline{Allen I. Mincer}{Professor}{
        allen.mincer@nyu.edu\newline
        ATLAS experiment, Department of Physics, New York University
    }
    \multiline{Uta Klein}{Professor}{
        uklein@hep.ph.liv.ac.uk\newline
        ATLAS experiment, Department of Physics, University of Liverpool

    }
    \multiline{Dr. Jeremy R.Love}{DOE}{
        jeremy.love.phd@gmail.com
        Former Physicist in ATLAS experiment, Argone National Lab
    }
    \multiline{Dr. Willilam Panduro Vazquez}{Staff scientist}{
        j.panduro.vazquez@cern.ch\newline
        ATLAS Experiment, Royal Holloway, U. of London
    }
    \multiline{Prof. Sara Strandberg}{Professor}{
        strandberg@fysik.su.se\newline
        Stockholm University, Sweden
    }
    \multiline{Walter Hopkins}{Assistant Physicist}{
         Walter.Hopkins@cern.ch\newline
         ATLAS Experiment, Argone National Lab

    }
    %\multiline{Kyle}{}{
    %}
\end{cventriesExtended}


%%%%%%%%%%%%%%%%%%%%%%%%%%%%%%%%%%%%%%%%%%%%%%%%%%%%%%%%%%%%%%%%%%%%%%%%%%%%%%
%%References
\newpage

\section{Main publications}

\begin{cventries}

\paper{2022}{Y. Abulaiti, Status of searches for dark matter at the LHC, ATL-PHYS-PROC-2022-003, Proceedings of the 32nd Rencontres de Blois on Particle Physics and Cosmology, Blois, Fr, 17 - 22 Oct 2021.}
    
\paper{2022}{ATLAS Collaboration, Search for heavy particles in the $b$-tagged di-jet mass distribution with additional $b$-tagged jets in proton-proton collisions at $\sqrt{s}=13$ TeV with the ATLAS experiment, Phys. Rev. D 105, 012001 (2022).}
	
\paper{2020}{ATLAS Collaboration, Search for new resonances in mass distributions of jet pairs using 139 fb$^{-1}$ of $pp$ collisions at $\sqrt{s}=13$ TeV with the ATLAS detector, JHEP 03 (2020) 145, 10.1007/JHEP03(2020)145.}
	
\paper{2019}{Y. Abulaiti, Search for new Resonances (lepton, jets) at ATLAS and CMS, Proceedings of the $54^{th}$ Rencontres de Moriond (2019) 133,  ATL-PHYS-PROC-2019-034.}
	
\paper{2016}{Y. Abulaiti, Search for top squark pair production in final states with one isolated lepton, jets, and missing transverse momentum in $\sqrt{s}=8$ TeV pp collisions with the {ATLAS} detector, Nuclear and Particle Physics Proceedings 273-275 (2016) 2415-2417.}
	
\paper{2016}{ATLAS Collaboration, Search for top squarks in final states with one isolated lepton, jets, and missing transverse momentum in $\sqrt{s} = 13$ TeV pp collisions with the ATLAS detector, Phys.Rev. D94 (2016) no.5, 052009, DOI:10.1103/PhysRevD.94.052009.}
	
\paper{2016}{Search for top squarks in final states with one isolated lepton, jets, and missing transverse momentum in $\sqrt{s} = 13$ TeV pp collisions with ATLAS data, ATLAS-CONF-2016-050.}

\paper{2015}{ATLAS Collaboration, ATLAS Run 1 searches for direct pair production of $3^{rd}$ generation squarks at the Large Hadron Collider, Eur. Phys. J. C75 no. 10, (2015) 510, arXiv:1506.08616 [hep-ex].}

\paper{2015}{ATLAS Collaboration, Summary of the ATLAS experiment's sensitivity to supersymmetry after LHC Run 1 interpreted in the phenomenological MSSM, JHEP 10 (2015) 134, arXiv:1508.06608 [hep-ex].}

\paper{2014}{ATLAS Collaboration, Search for top squark pair production in final states with one isolated lepton, jets,  and missing transverse momentum in $\sqrt{s}=8$~TeV pp collisions with the ATLAS detector, JHEP 1411 (2014) 118, DOI: 10.1007/JHEP11(2014)118.}

\end{cventries}


\end{document}
